\section{Experiment Design}\label{sec:experiment}

The most basic meaningful interaction task with a display is to place a cursor
on a target region and ``click'' it to indicate activation intention.  Since
this study was intended to evaluate the differences in input/output
modalities, we limited it to the most basic ``point and click'' operation
possible: two targets are displayed on the screen at random points along with
a cursor.  The \emph{start} target is initially red and when the user
successfully clicks on it, it turns grey and the \emph{end} target becomes
red.  In the following we refer to each pair of start and end clicks for a
given input/output combination as a \emph{trial}.

Even though real world applications that deal with 3D data often involve dense
scenes with occlusion, it is the authors' anecdotal experience that scenes are
transformed (rotated and translated) by the user prior to interaction in order
to make the desired interaction points congruent with the plane of the screen.
This is equivalent to aligning the axes of the points' principle components
with the screen, thus maximizing the accuracy of an input device and the
projected resolution in pixels.  Therefor, we limited the target points to be
randomly generated near the surface of a truncated sphere, such that they were
approximately aligned to the plane of the screen.  This means there is no
occlusion and that the euclidean distance between target points is close to
constant across all trials.  Randomizing target locations is employed to prevent
motor learning effects as a confound.

We recruited 13 individuals to test on the platform described in \ref{design},
of which 11 provided suitable data: the first subject revealed flaws in our
test setup, invalidating his results, and another did not complete the study.
All of the subjects except one are either current students or professoinals in
STEM fields.

Each subject was given a pre-experiment survey (summarized in Table \ref{}) to
ascertain prior familiarity with 3D systems and interaction, and a
post-experiment survey (summarized in Table \ref{}) to provide feedback on the
performance and feel of our test system.  From the pre-survey we see that even
thoug all the subjects are well familiar with computing systems, that their
level of experience with 3D systems and interaction uniformly ranges the gamut
from little-to-none (e.g. has seen the occasional 3D movie or played a video
game set in a 3D world) to highly (e.g. has much experience in 3D design).

This study was a within-subjects 3x2 design where the input factor has levels
\{Mouse and Keyboard, 3D Mouse, Leap\} and the output factor has levels \{2D
Projections, 3D Headtracked\}.  Each subject was asked to complete 20 trials
for each of the six input/output combination, divided into 12 sessions of 10
trials each.  We used a Latin Square to vary the order of the 12 sessions for
each subject to minimize learning effects from one input/output modality
affecting another.  Each subject was given 2 trial tutorial for each modality
to familiarize them with how to operate our system prior to starting the full
120 trials.  Subjects were not required to complete all trials and were
informed that they could stop at any time, though only one chose to stop early
after completing a round of 60 trials due to time constraints.  Total time
spent per subject including answering the surveys was around 40 minutes.

We logged cursor pose (3D position and orientation) for each trial at 30Hz for
the duration between clicks on the start and end targets.

