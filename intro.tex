\section{Introduction}\label{sec:intro}
Humans operate in a world of three spatial dimensions, and have developed a
corresponding intuition on interacting with physical objects.
If we see a coffee mug on a table, we can reach out and grab the mug with
absolute confidence in where our hand is going
Unfortunately, when dealing with virtual 3D data, most of this intuition is
lost. Computer-aided design systems and 3D modelling software present 3D scenes
on 2D screens and require the use of a keyboard and inherently
2-degree-of-freedom mouse to perform such complicated tasks as selection,
rotation, and viewpoint control. This results in complicated, unintuitive
workflows; for example, the standard method to move an object in a virtual scene
is to decompose a translation into several planar movements, usually aided by
orthogonal projections (front view, side view, top view). In this paper we
investigate methods to make interacting with virtual 3D data as fluid and
natural as interacting with the physical world.

We focus our efforts on examining the input and output devices that are
currently used in 3D user interfaces. Several existing systems have already
attempted to make use of physical intuition by developing high degree-of-freedom
input devices to interact with 3D information \cite{manders2010gesture,
mattheiss2011navigating,mine1997moving}, or depth cues such as
stereoscopy or parallax \cite{boritz1997study, schultheis2012comparison}.
Many of these works include studies showing the superiority (or inferiority)
of the system compared with traditional mouse-and-keyboard systems.

However, these studies focus on only half of the factors relevant for intuition;
visual feedback is rarely adapted to react seamlessly with the input device.
Instead, natural input devices are used to manipulate an onscreen object, which
is very different from intuitively reaching out to the object. In our coffee
mug example, we have to manipulate the mug at the end of a long stick. We
believe that by colocating the 3D interaction space with a 3D display space
with appropriate depth cues, we can recover the intuitive feeling of interacting
with a physical object. Imagine that your mug, physically located on the table
on the other side of the room, was virtually rendered directly in front of you;
you can once again reach out and "grab" it intuitively, as if it was real.

In this paper, we present the results of a preliminary study investigating the
effects of both input and output methods on performance in a 3D point selection
task. Specifically, we compare the standard 2D monitor with orthographic
projections with a stereoscopic, head-tracked output display. We also compare a
standard mouse and keyboard input with a 6-DOF ``3D mouse'' and a free-space
input sensor with a 1-to-1 mapping to the virtual space.
