\begin{abstract}
3D interaction using the traditional mouse, keyboard and 2D screen is often
challenging and unintuitive, due to the inherently 2D nature of these devices.
Many previous systems have examined the effects of higher degree-of-freedom
input devices and enhanced output devices on 3D task performance. However very
few have investigated the effects of colocating input spaces with perceived
output spaces, which we hypothesize will maximize the use of spatial intuition.
We present a preliminary study that compares performance on a 3D
point selection task across different input modalities (traditional mouse and
keyboard, a standard 6-DOF wand device, and a colocated free-space method) and
output modalities (traditional 2D display and a stereoscopic, head-tracked 3D
display). The results indicate the superiority of stereoscopic output and
suggest that, while current 3D input technology is still immature, high
degree-of-freedom input devices are much more effective than the mouse and
keyboard.
\end{abstract}
