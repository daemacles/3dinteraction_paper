\section{Discussion and Future Work}\label{sec:discussion}

Given the uniform distribution over prior 3D experience, we feel that our
results generalize at least to anyone with familiarity with computers.

Anecdote from \cite{leewii} about how parallax is really pretty immersing,
even if it is only approximately calibrated to head position.  Also the fact
that human ``steo'' vision is basically flat beyond 20 feet.  So, stereoscopic
displays are really only useful when interacting with objects at close range,
i.e. basically within arms reach, as our study did.

1-to-1 mapping as we implemented it requires a lot of setup (calibration process) and once done the components can't be moved.  This limits its ease of adoption in informal settings / at home / on laptops with 3d displays.

\subsection{User Feedback}\label{sec:feedback}

An important aspect of this experiment was to gather user feedback about their
impressions in using each modality; the most efficient interaction modality is
useless to implement if users dislike using it.  Feedback is also used to
suggest improvements to our system for future, larger scale testing.

On the whole, user feedback about our test system was very positive.  There
was very much a sense of the Wow! factor when using the full 3D systems,
despite its virtual wack-a-mole aspect, and several subjects commented that it
was fun in and of itself.

In the 3D output mode, subjects vastly preferred using the Hydra to the Leap
or mouse, even though the majority found the Leap to be the most intuitive
input, see \figref{fig:post}.  Based on their comments this is mainly due to
two factors: first, the Leap can be frustrating to use due to its imperfect
tracking (i.e. it either works well or horribly, no middle ground), and
second, that shoulder fatigure rapidly sets in, limiting the time for
comfortable interaction.  Another stated benefit of the Hydra was that its
relative motion allowed a user to reposition her hand without interrupting the
state of the workflow.

\begin{itemize}
\item ``[The Hydra] felt the most intuitive, and felt like it has the most
  potential for me to increase speed/precision over use.  The leap motion with
  3D viewing was a close second, but the fact that \emph{my hand occludes the
  display} during use was not too comfortable.''
\item ``Because [the Hydra] provides more accurate control and doesn't require
  too much effort to move. Leap motion is a pain for both cases, because it is
  always out of the detection range and it requires my arm to point to a
  precise location that is more tiring.''
\item ``The 3D headtracked monitor helped me see how close the object was to
  me. I liked the relative motion of the [Hydra]because it allowed me to
  reposition my hand and keep my arm in a comfortable range of motion.''
\end{itemize}

When asked about improvements to our system, a clear consensus was replacing the Leap with something more reliable.  Other suggested improvements involve the
UI design (include more graphical cues such as axes or cursor location
indicators) and improved 3D realism such as shading and shadows that
correspond with the physical environment.

\begin{figure}
    \centering
    \begin{tabular}{c | c | c}
    Favorite Combination & 2D & 3D \\ \hline
    Mouse and Keyboard   &    & 1 \\
    Hydra                &    & 9 \\
    Leap Motion          &    & 1 \\
    \end{tabular}

    \vspace{0.2in}

    \begin{tabular}{c | c | c}
    Most Intuitive     & 2D & 3D \\ \hline
    Mouse and Keyboard &    &  \\
    Hydra              &    & 4 \\
    Leap Motion        &    & 7 \\
    \end{tabular}

    \caption{Post-experiment responses}
    \label{fig:post}
\end{figure}

