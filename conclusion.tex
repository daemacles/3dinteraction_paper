\section{Conclusion}\label{sec:conclusion}
In this paper we evaluated the effects of different input and output methods on
performance in a 3D task. We specifically investigated methods that leveraged
human physical intuition, focusing on overlapping the visual output space with
a physical input space.  We compared the current standard 2D monitor output
modality with a stereoscopic, head-tracked 3D monitor output method that
displayed virtual scenes in real space, in front of the monitor. We also
compared the mouse and keyboard interface typically used in current 3D modelling
systems with two high degree-of-freedom input methods: a wand-based 3D analog of
the mouse, and a free-space in-air system that was aligned with the output
space.

To compare 3D user interfaces, we performed an experiment where users completed
several 3D placement tasks using each input/output combination. Our results
showed that stereoscopic output resulted in superior performance to 2D outputs,
and was unanimously preferred by all of our users. High degree-of-freedom input
devices had lower median completion times, and the fastest interface combination
was the wand device with the stereoscopic output. Informal analysis of
trajectories and learning effects within a single subject-input-output
combination suggests that these high degree-of-freedom devices are also more
intuitive to use, and post-experiment surveys showed that they were also
preferred by most users.

These results show that interaction modalities that more closely map to the
3D nature of the task provide clear benefits over standard 2D interfaces. We
will use the insights gained from this work to inform the design of a larger
study, with a more refined system design, to gain more powerful conclusions.
These conclusions can help inform the design of 3D systems for many
applications, including improving animation workflows, enhancing scientific
visualizations, and aiding design processes. We hope to make virtual 3D
interaction for all of these applications as intuitive and natural as
our interactions with the physical world.
