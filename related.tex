\section{Related Work}\label{sec:related}
Mixed reality systems often focus on the human hand for natural interaction.
Although several 3D manipulation systems use direct mappings between hand
position and a virtual hand, such as \cite{poupyrev1996go},
comparatively few aim to make the perceived 3D position of the hand proxy be
coincident with the user's natural perception of their
hand location. The relevant perceptual phenomenon, known as {\bf
proprioception}, is a person's automatic knowledge of the position and
orientation of their
limbs relative to themselves. The use of proprioception in 3D interfaces was
investigated by Mine in \cite{mine1997exploiting}, which found that
in an immersive virtual environment, manipulating objects colocated with the
users hand was more efficient than manipulation at an offset. Several systems,
such as \cite{mulder2002personal, prachyabrued2011dropping}, use a two-layered
tabletop with a mirrored display above a physical table; in these systems the
interaction space, underneath
the display and above the table, is directly mapped to what the user sees in the
display. The aforementioned systems are strictly virtual reality systems, since
the actual environments displayed
are still completely synthesized. Recent work by Microsoft Research, the
Holodesk \cite{holodesk}, has instead used half-silvered mirrors to create an
augmented reality
interaction space, where the user can see their hands interacting in the same
world as virtual objects. Holodesk focused on realistic physical interactions
within the space,
representing real objects as point clouds that can knock over or hold up other
objects.

Several works have investigated the effectiveness of novel, high-DOF input
devices for 3D tasks by directly comparing them to traditional devices,
especially the mouse.
Evaluation methods generally involve performing fundamental 3D tasks, especially
selection, placement, and rotation. Input methods are compared based on their
speed and accuracy. The SpaceNavigator, a commercially available high DOF
device, was found to be inferior to the mouse for placement tasks in
\cite{mattheiss2011navigating}. B{\'e}rard \etal also found the mouse to be more
effective than several high-DOF devices for placement tasks
in \cite{study1}.  However, a later study by the same group found that enhanced
visual feedback, in the form of pop-up depth views, greatly enhanced the
effectiveness
of the high-DOF device, and was found to perform better than the mouse
\cite{study2}. Schultheis \etal find the mouse inferior to high DOF input devices
for a docking task consisting of achieving a target placement and orientation
\cite{schultheis2012comparison}.

There have also been works examining the effects of depth cues on 3D task
performance. Wanger \etal examine 2D renderable depth-cues, such as shadows,
background textures, and motion cues, in \cite{wanger1992perceiving}. Boritz and
Booth compare monoscopic and stereoscopic displays, as well as viewpoint
tracking, with subjects using a 6-DOF input device, finding stereoscopic
displays to be superior to monoscopic displays \cite{boritz1997study}. In
contrast, Schultheis \etal found no significant
difference between monoscopic and stereoscopic outputs
\cite{schultheis2012comparison}. The evaluation of the Holodesk
system also involved
a user study, in which they compared performance on a selection task with varying
depth cues. This study
focused on the effectiveness of aligning the location of physical objects in the tabletop
space with the apparent locations of virtual objects. They found that aligned
stereoscopic display systems were better than aligned non-stereoscopic systems
with appropriate occlusion cues, and unaligned displays resulted in the slowest
performance \cite{holodesk}. As far as we are aware, only Schultheis \etal have
simultaneously varied 3D input and output modalities; however they did not
examine direct colocation of input and output spaces as Holodesk did
\cite{schultheis2012comparison}.

